% Options for packages loaded elsewhere
\PassOptionsToPackage{unicode}{hyperref}
\PassOptionsToPackage{hyphens}{url}
%
\documentclass[
]{article}
\usepackage{amsmath,amssymb}
\usepackage{lmodern}
\usepackage{iftex}
\ifPDFTeX
  \usepackage[T1]{fontenc}
  \usepackage[utf8]{inputenc}
  \usepackage{textcomp} % provide euro and other symbols
\else % if luatex or xetex
  \usepackage{unicode-math} % this also loads fontspec
  \defaultfontfeatures{Scale=MatchLowercase}
  \defaultfontfeatures[\rmfamily]{Ligatures=TeX,Scale=1}
\fi
\usepackage{lmodern}
\ifPDFTeX\else
  % xetex/luatex font selection
\fi
% Use upquote if available, for straight quotes in verbatim environments
\IfFileExists{upquote.sty}{\usepackage{upquote}}{}
\IfFileExists{microtype.sty}{% use microtype if available
  \usepackage[]{microtype}
  \UseMicrotypeSet[protrusion]{basicmath} % disable protrusion for tt fonts
}{}
\makeatletter
\@ifundefined{KOMAClassName}{% if non-KOMA class
  \IfFileExists{parskip.sty}{%
    \usepackage{parskip}
  }{% else
    \setlength{\parindent}{0pt}
    \setlength{\parskip}{6pt plus 2pt minus 1pt}}
}{% if KOMA class
  \KOMAoptions{parskip=half}}
\makeatother
\usepackage{xcolor}
\usepackage{graphicx}
\usepackage[margin=1in]{geometry}
\setlength{\emergencystretch}{3em} % prevent overfull lines
\providecommand{\tightlist}{%
  \setlength{\itemsep}{0pt}\setlength{\parskip}{0pt}}
\setcounter{secnumdepth}{-\maxdimen} % remove section numbering
\newlength{\cslhangindent}
\setlength{\cslhangindent}{1.5em}
\newlength{\csllabelwidth}
\setlength{\csllabelwidth}{3em}
\newlength{\cslentryspacingunit} % times entry-spacing
\setlength{\cslentryspacingunit}{\parskip}
\newenvironment{CSLReferences}[2] % #1 hanging-ident, #2 entry spacing
 {% don't indent paragraphs
  \setlength{\parindent}{0pt}
  % turn on hanging indent if param 1 is 1
  \ifodd #1
  \let\oldpar\par
  \def\par{\hangindent=\cslhangindent\oldpar}
  \fi
  % set entry spacing
  \setlength{\parskip}{#2\cslentryspacingunit}
 }%
 {}
\usepackage{calc}
\newcommand{\CSLBlock}[1]{#1\hfill\break}
\newcommand{\CSLLeftMargin}[1]{\parbox[t]{\csllabelwidth}{#1}}
\newcommand{\CSLRightInline}[1]{\parbox[t]{\linewidth - \csllabelwidth}{#1}\break}
\newcommand{\CSLIndent}[1]{\hspace{\cslhangindent}#1}
\usepackage{subfig}
\usepackage{wrapfig}
\usepackage{longtable}
\ifLuaTeX
  \usepackage{selnolig}  % disable illegal ligatures
\fi
\IfFileExists{bookmark.sty}{\usepackage{bookmark}}{\usepackage{hyperref}}
\IfFileExists{xurl.sty}{\usepackage{xurl}}{} % add URL line breaks if available
\urlstyle{same}
\hypersetup{
  pdftitle={Comparing and Intereting Machine Learning Algorithms Estimating Technical Prices},
  pdfkeywords={mlr3, machine learning, regression, non-life insurance,
estimating technical price, XGBoost, debiassing, bias, SHAP-values,
interpretation of ML models},
  hidelinks,
  pdfcreator={LaTeX via pandoc}}

\title{Comparing and Intereting Machine Learning Algorithms Estimating
Technical Prices}
\author{true}
\date{2023-05-08}

\begin{document}


\Large\textbf{Comparing and Intereting Machine Learning Algorithms
Estimating Technical Prices}
\normalsize

\noindent \textbf{ Joakim Bilyk, Sebastian Cramer and Teis
Blem} \hfill  \emph{\small University of Copenhagen}   


\hbox{\vrule height .2pt width 39.14pc}

\noindent This document provides a practical example of the application
of supervised machine learning algorithms to car insurance data using
the mlr3 package. In non-life insurance pricing, a popular model is the
frequency-severity model, which decomposes the price into the product of
the probability of a claim arising and the expected claim size given a
claim occurs. This paper argues that a tree-based model is well-suited
to the frequency-severity model, as it can capture complex nonlinear
relationships between risk factors and claims. To interpret the model's
estimates, we used shapley values to gain insights into the relative
importance of each risk factor. Finally, we used a decomposition
technique to debias the price model and ensure it does not discriminate
based on gender. Overall, our approach demonstrates the potential of
machine learning to create more accurate and equitable pricing models in
the insurance industry.


\noindent \emph{Keywords}: mlr3, machine learning, regression, non-life
insurance, estimating technical price, XGBoost, debiassing, bias,
SHAP-values, interpretation of ML models \par

 \hbox{\vrule height .2pt width 39.14pc}




{
\hypersetup{linkcolor=black}
\setcounter{tocdepth}{2}
\tableofcontents
}
\newpage

\hypertarget{getting-familiar-with-the-data}{%
\section{Getting familiar with the
data}\label{getting-familiar-with-the-data}}

dsad (Lützen 2019)

\hypertarget{refs}{}
\begin{CSLReferences}{1}{0}
\leavevmode\vadjust pre{\hypertarget{ref-lutzen2019}{}}%
Lützen, Jesper. 2019. \emph{Diskrete Matematiske Metoder}. 2nd ed.
Copenhagen, Denmark.

\end{CSLReferences}

\end{document}